\chapter{Introduction}

\section{Numeric kinds}

Following the portable programming paradigms used in current Fortran 95,
\qmcpack{} defines its own numeric kinds in the \verb!numeric_kinds! module.
This module provides the following four integer kinds:
\begin{center}
\begin{tabular}{cc}
Integer kind & Maximum value \\ \hline
\verb!i1b! & $10^2$ \\
\verb!i2b! & $10^4$ \\
\verb!i4b! & $10^9$ \\
\verb!i8b! & $10^{18}$ \\ \hline
\end{tabular}
\end{center}
%
The following real kinds are also supported:
%
\begin{center}
\begin{tabular}{cc}
Real kind & precision \\ \hline
\verb!sp! & single    \\
\verb!dp! & double    \\
\verb!qp! & quadruple if available, double otherwise \\ \hline
\end{tabular}
\end{center}
%
Note that for the integer kinds, most combinations of architecture and compiler
will support \verb!i1b!, \verb!i2b! and \verb!i4b! but \verb!i8b!
might not always be available.

The real kinds are special in the sence that the concepts of `single', `double'
and `quadruple' precision should not be associated with byte-lenghts but must 
be interpreted as \emph{abstract} concepts of precision.  The only rules that
apply are that `single' precision is the least significant type of reals on an
architecture.  Variables of `double' precision will have twice the decimal
precision of a `single precision' variable and `quadruple precision'
variables will have twice the precision of a `double' precision variable.
If `quadruple' precision is not available, variables of type \verb!qp! will
have the same precision as variables of type \verb!dp!.

To see what kinds are available for your combination of architecture and
compiler, the program \texttt{test\_numeric\_kinds} located under the
\texttt{tests} directory can be used.

\section{Points and functions} \label{sec:points_functions}

An $s$-dimensional point in \qmcpack{} is represented by a rank one
array, being either of fixed size 
%
\begin{lstlisting}
real(kind=qp), dimension(5) :: x
...
x = (/ 0.1_qp, 0.2_qp, 0.3_qp, 0.4_qp, 0.5_qp /)
\end{lstlisting}
%
or being an allocatable array
%
\begin{lstlisting}
real(kind=qp), dimension(:), allocatable :: x
...
allocate(x(5))
x = (/ 0.1_qp, 0.2_qp, 0.3_qp, 0.4_qp, 0.5_qp /)
...
deallocate(x)
\end{lstlisting}
%
Functions in \qmcpack{} can have parameters and can be divided into two groups:
\begin{itemize}
\item functions acting on a single $s$-dimensional point, which return one single value for the function evaluated at that point.
\item functions acting on $N$ $s$-dimensional points at the same time and which
return a rank one array of length $N$ containing the function values at those
points.
\end{itemize}
%
The first type of functions follows the naming convention
\begin{lstlisting}
f_name_1p(x, params) result (y)
\end{lstlisting}
where \verb!x! is a one-dimensional array of kind \verb!qp! and \verb!y! is a
single variable of kind \verb!qp!.  The second type of functions have the
naming convention
\begin{lstlisting}
f_name_np(x, params) result (y)
\end{lstlisting}
where \verb!x! is now an $s \times N$ dimensional array.  Important to note
is that the \emph{rows} of \verb!x! represent the different \emph{dimensions}
for the pointset \verb!x!!

Both types of functions can take parameters.  The parameters given as an
argument to a function must be of the derived-type
\begin{lstlisting}
type, public :: functionparams
  real(kind=qp), dimension(:), pointer :: a
  real(kind=qp), dimension(:), pointer :: c
  real(kind=qp), dimension(:), pointer :: u
end type functionparams
\end{lstlisting}
%
Before variables of this type can be passed as parameters to functions, memory
for them must be allocated and certain parameters must be assigned a value with
\begin{lstlisting}
call allocate_function_parameters(params, s)
\end{lstlisting}
Assuming \verb!x! is a single three-dimensional point, the typical way of
allocating memory and assigning values to the parameters is as follows:
\begin{lstlisting}
real(kind=qp), dimension(3) :: x
real(kind=qp)               :: y
type(functionparams)        :: params
...
s = size(x)
call allocate_function_parameters(params, s)
...
params%a = (/ 0.1_qp, 0.2_qp, 0.3_qp /)
params%u = (/ 0.3_qp, 0.2_qp, 0.1_qp /)
...
y = f_name_1p(x, params)
\end{lstlisting}
Note that of course, only the parameters being used for the function evaluation
must be assigned a value.  Functions that don't take parameters still must have
a dummy argument \verb!params!.
