\chapter{\qmcpack{} installation}

\section{System requirements}

To be able to compile and use \qmcpack{}, the following is required:

\begin{itemize}
\item The \texttt{makedepf90} program must be somewhere in your execution path.
      \texttt{makedepf90} is obtainable from
      \begin{quote}
      \url{http://www.helsinki.fi/~eedelman/makedepf90.html}
      \end{quote}
\item A working Fortran 90/95 compiler should be present.  \qmcpack{} has been
      tested and should work with at least the following compiler-versions:
      \begin{center}
       \begin{tabular}{|ccl|} \hline
          compiler               & most recent tested version & website \\ \hline
          G95                    & GCC 4.0.1 (g95!) Mar 11 2006 & \url{http://www.g95.org/} \\
          GNU Fortran 95         & GCC 4.1.0 20050907 & \url{http://gcc.gnu.org/fortran/} \\
          NAGWare f95            & Release 5.0 (Edit 414) & \url{http://www.nag.com/} \\
          F                      & Release 20031017 & \url{http://www.fortran.com/F/} \\
          Intel Fortran Compiler & Version 9.0, Build 20050430 & \url{http://www.intel.com/} \\ \hline
       \end{tabular}
      \end{center}
      If it doesn't, please send us a bug report.

\item The \verb!make! program to build the whole project.
\end{itemize}

\section{Installing \qmcpack{}}

After unpacking the \verb!.tar.gz! file, change into the top-level directory
and open the \verb|make.inc| file.  Edit this file to your own needs, it should
be self-explanatory what to change.  The most important variables to set are
\verb!FC! for the Fortran 95 compiler to use, \verb!MODE! which can be either
\verb!debug! or \verb!optimized!, and \verb!INSTALLDIR! which is the location
where the library and module files should be installed.

Once that is done, a simple
\begin{verbatim}
make
\end{verbatim}
will build the library, the tests and the timings using the default compiler
specified at the top of the \verb!make.inc! file.
If you only want to build the library, then a simple
\begin{verbatim}
make qmclib
\end{verbatim}
should be enough.  If you only want to build the tests or the timings, then
simply use
\begin{verbatim}
make tests
\end{verbatim}
or
\begin{verbatim}
make timings
\end{verbatim}
To clean up everything, type
\begin{verbatim}
make clean
\end{verbatim}
or
\begin{verbatim}
make realclean
\end{verbatim}
The \verb!clean! target is a bit less agressive than the \verb!realclean!
target.

To use the \qmcpack{} library for your own programs, simply add the appropriate
USE-statement to your Fortran program:
\begin{verbatim}
use qmcpack
\end{verbatim}
and make sure that the \qmcpack{} installation directory that you specified in
the \verb!make.inc! file is added to the module and
library search path of your Fortran compiler.  How to add these directories to the appropriate paths is
compiler-dependent.  See the manual of your compiler on how to do this.

All of the above \verb!make!-commands will use the default compiler and
compile-mode as specified in the \verb!make.inc! file.
Compiling with another compiler can simply be done by overriding the \verb!FC!
makefile-variable.
\begin{verbatim}
make FC=<F|g95|ifc|ifort|gfortran>
\end{verbatim}
where you should choose one of the compilers specified between the brackets.
The compile-mode can be specified by overriding the \verb!MODE!
makefile-variable:
\begin{verbatim}
make MODE=<debug|optimized|profile>
\end{verbatim}
