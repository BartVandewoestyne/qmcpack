\chapter{Low-discrepancy sequences}

\qmcpack{} can generate several types of low-discrepancy sequences.  The general
procedure for generating low-discrepancy sequences is as follows:
\begin{description}
\item[Step 1:] Initialize the generator of the pointset by setting some
 pointset-specific values.
\item[Step 2:] Call either one of the `sequential' or `block'-methods to generate points.
 A call to a sequential methods always returns the \emph{next point} in the
 sequence while a call to a block-method immediately returns a \emph{whole set}
 of points.
\item[Step 3:] Use your point(s) for whatever you want to use them.
\item[Step 4:] When done, free up the memory consumed by one the
  pointset-specific variables that were initialized in step 1.
\end{description}
%
\qmcpack{} currently supports the following types of sequences:
%
\begin{itemize}
\item Weyl sequences
\item Haber sequences
\item Faure sequences
\item Halton sequences
\item Hammersley point set
\item Latin Hypercube Samples
\item Sobol sequences
\end{itemize}
%
In the following sections, the definition of each type of sequence is given and
it is explained how to generate points from the sequence using the \qmcpack{} library.

\section{Weyl sequences}

The $i$th point of the Weyl sequence \cite{weyl16} is defined as
\[
x_i = ( \{i\theta_1\}, \{i\theta_2\}, \ldots, \{i\theta_s\} ),
 \qquad i=1,2,3,\dots
\]
where $\{x\}$ denotes the fractional part of $x$ and all the $\theta_1,
\theta_2,\ldots,\theta_s$ are independent irrational numbers.
The initialization of the Weyl-sequence generator within \qmcpack{} is done with
the command
\begin{lstlisting}
call init_weyl(s, irrationals [, n_start] [, step])
\end{lstlisting}
Here, \verb!s! is the dimension of the sequence and \verb!irrationals! is an
\verb!s!-dimensional vector representing $\theta_1, \theta_2,\dots,\theta_s$.
\verb!n_start! is an optional dummy argument that allows the user to set an
initial value for the index $i$ for each dimension.  \verb!n_step! then
determines how much the index is being increased at each generation of the next
point.  This means \qmcpack{} allows generation of Weyl sequences of the form
\[
x_i = ( \{(\verb!n_start(1)!+\verb!step(1)!\times i)\theta_1\}, \ldots, \{(\verb!n_start(s)!+\verb!step(s)!\times i)\theta_s\}),
\]
for $i=1,2,3,\ldots$  Successive points of the Weyl-sequence can be generated with the command
\begin{lstlisting}
call next_weyl(x)
\end{lstlisting}
%
After finishing the computations, memory used by the internals of the Weyl
sequence generator can be freed up with
\begin{lstlisting}
call free_weyl()
\end{lstlisting}
%
Depending on the choice of the irrationals, different `instances' of the
Weyl-sequence are known in the literature.  The following subsections explain
which ones are supported by \qmcpack{}.

\subsection{Square-root sequences}

Taking the $s$ irrationals as the square roots of prime numbers
$p_1, \ldots, p_s$, one gets the sequence
\[
x_i = ( \{i\sqrt{p_1}\}, \{i\sqrt{p_2}\}, \ldots, \{i\sqrt{p_s}\} ),
 \qquad i=1,2,3,\ldots
\]
which can be initialized in \qmcpack{} using
\begin{lstlisting}
call init_square_root_seq(s [, myprimes] [, n_start] [, step])
\end{lstlisting}
where \verb!n_start! and \verb!n_step! have the same meaning as in the
generation of the more general Weyl sequences and \verb!myprimes! is an optional
\verb!s!-dimensional vector containing the primes $p_1, \ldots, p_s$ to be used.
If no primes are specified, the first $s$ primes are taken by default.

Subsequent points of the square-root sequence can then be generated with
\begin{lstlisting}
call next_square_root(x)
\end{lstlisting}

\subsection{gp sets}

Fang and Wang \cite{fang94wang} mention another set of irrationals for what
they call `good point sets' (gp sets).  Let $p$ be a prime and $q=p^{1/(s+1)}$,
then one of the gp sets they mention consists of points
\[
x_i = ( \{iq\}, \{iq^2\}, \dots, \{iq^s\} ), \qquad i=1,2,3,\dots
\]
Initialization within \qmcpack{} for this set is done with
\begin{lstlisting}
call init_gp_set_b(s, p [, n_start] [, step])
\end{lstlisting}
where the scalar variable \verb!p! is the prime to be used, and the other
parameters are as usual for Weyl sequences.  Successive points can be
generated with
\begin{lstlisting}
call next_gp_set_b(x)
\end{lstlisting}

The \emph{cyclotomic field method} also appearing in Fang and Wang's book
lists another set of irrationals and leads to a sequence of the form
\[
x_i = \left( \left\{i\left\{2\cos{\frac{2\pi}{p}}\right\}\right\}, \left\{i\left\{2\cos{\frac{4\pi}{p}}\right\}\right\}, \ldots, \left\{i\left\{2\cos{\frac{2\pi s}{p}}\right\}\right\} \right),
 \qquad i=1,2,3,\ldots
\]
These sequences can be initialized using
\begin{lstlisting}
call init_gp_set_c(s, p [, n_start] [, step])
\end{lstlisting}
and subsequent points are to be generated with
\begin{lstlisting}
call next_gp_set_c(x)
\end{lstlisting}

\section{Haber sequences}

The $i$th point of the Haber sequence \cite{shaw88, fang94wang} is defined as
\[
x_i = \left( \frac{i(i+1)}{2}\sqrt{p_1},\ldots, \frac{i(i+1)}{2}\sqrt{p_s}
\right) \pmod{\vec{1}},
\]
where the $p_j$ are prime (usually $p_j$ is the $j$th prime).  Before using any routine related to the Haber sequence, it must be initialized
with
\begin{lstlisting}
call init_haber(s [, startindex] [, myprimes])
\end{lstlisting}
Here, \verb!s! is the dimension of the sequence and \verb!startindex! and
\verb!myprimes! are optional rank one
array arguments of length \verb!s! containing startindices $i_1,\ldots,i_s$
to start from for each dimension and primes $p_1,\ldots,p_s$.  If no startindices are specified, \verb!(/ 1,..., 1 /)! is taken as the startindex.  If no primes are specified, the first $s$ primes are taken.
Subsequent Haber points can then be generated using the subroutine
\begin{lstlisting}
call next_haber(x)
\end{lstlisting}
or it is possible to generate a set of \verb!n! \verb!s!-dimensional Haber
points using
\begin{lstlisting}
call haber(n, s, startindex, myprimes, x)
\end{lstlisting}
where \verb!x! will contain the \verb!n! \verb!s!-dimensional points after the
call, and \verb!startindex! and \verb!myprimes! are as previously defined.
After all calls to \verb!next_haber(x)! are done, memory has to be freed
using
\begin{lstlisting}
call free_haber()
\end{lstlisting}

\section{Radical inverse-functionality}

Let $b \geq 2$ be an integer, then any integer $n \geq 0$ can be written in the form
\[
 n = d_0 + d_1b + d_2b^2 + \cdots + d_jb^j, \qquad 0 \leq d_i < b.
\]
The radical inverse function $\phi_b(n)$ for base $b$ is defined by
\begin{equation} \label{eq:radical_inverse_function}
\phi_b(n) = \frac{d_0}{b} + \frac{d_1}{b^2} + \cdots + \frac{d_j}{b^{j+1}}.
\end{equation}
To get the radical inverse value of a certain integer \verb!n! in a certain
integer base \verb!b! simply write
\begin{lstlisting}
phi = radical_inverse(n, b)
\end{lstlisting}
where \verb!phi! is a real of kind \verb!qp!.

Several modifications of the radical inverse function exist.  Most of these
change the digits $d_0,\ldots,d_j$ in a certain way so that another radical
inverse value is obtained.  Among the first modifications is Warnock's
\emph{folded radical inverse} function\cite{warnock72}
\[
\psi_b(n) = \frac{(d_0+0) \bmod b}{b} + \frac{(d_1+1) \bmod b}{b^2} + \cdots + \frac{(d_j+j) \bmod b}{b^{j+1}} + \cdots
\]
which can be obtained with
\begin{lstlisting}
phi = radical_inverse_folded(n, b)
\end{lstlisting}
%
A more general form of modification is the scrambled radical inverse function
\[
S_b(n) = \frac{\pi_b(d_0)}{b} + \frac{\pi_b(d_1)}{b^2} + \cdots + \frac{\pi_b(d_j)}{b^{j+1}}.
\]
which was first introduced by Braaten and Weller \cite{braaten79weller} and can
be called via
\begin{lstlisting}
phi = radical_inverse_scrambled(n, b, p)
\end{lstlisting}
Here, \verb!p! is a rank-1 array of length \verb!b! representing a certain
permutation of the digits $0,\ldots,b-1$.

To scramble all digits of the radical inverse function with a linear scrambling
of the form
\[
f(x) = wx+c \bmod{b}
\]
one calls
\begin{lstlisting}
phi = radical_inverse_lin_scrambled(n, b, w, c)
\end{lstlisting}
Here, \verb!n! and \verb!b! are as usual, and \verb!w! and \verb!c! are scalar
integers representing the parameters for the linear scrambling.

Taking a closer look at \cite{atanassov04}, we see that Atanassov's
\emph{modified radical inverse function} actually has 4 forms.  Depending on
whether one
uses $k_i^{j}$ or $k_i^{j+1}$ for the linear multiplication factor, and whether
one digit-scrambles the function with a randomly selected $b_j^{(i)}$ or not,
four types of linear digit scrambling can be derived:
\begin{enumerate}
\item The non-increased and non-digit scrambled version:
\begin{equation}
\pi_{p_i}(d_j) = d_jk_i^j \bmod p_i \label{eq:ni_ns}
\end{equation}
%
\item The increased and non-digit scrambled version:
\begin{equation}
\pi_{p_i}(d_j) = d_jk_i^{(j+1)} \bmod p_i \label{eq:i_ns}
\end{equation}
%
\item The non-increased and digit scrambled version:
\begin{equation}
\pi_{p_i}(d_j) = d_jk_i^j + b_j^{(i)} \bmod p_i \label{eq:ni_s}
\end{equation}
%
\item The increased and digit scrambled version:
\begin{equation}
\pi_{p_i}(d_j) = d_jk_i^{(j+1)} + b_j^{(i)} \bmod p_i \label{eq:i_s}
\end{equation}
\end{enumerate}
In \qmcpack{}, all this can be accomplished using
\begin{lstlisting}
call radical_inverse_modified(n, b, k [, increased] [, scrambled], radinv)
\end{lstlisting}
Here, \verb!n! and \verb!b! are as usual, and \verb!increased! and
\verb!scrambled! are two optional logical arguments indicating whether
the power $k^j$ (non-increased) or $k^{j+1}$ (increased) should be used and
whether digit scrambling should be applied or not.

A last type of radical inverse function which is described in
\cite{warnock95} combines the initial behavior of the Weyl sequence
\cite{weyl16} with the asymptotic behavior of the Halton sequence to construct
what Warnock calls the \emph{PhiCf sequence}.  Each $d_i$ in
(\ref{eq:radical_inverse_function}) is replaced with $S(b)d_i \bmod{b}$ to obtain a new kind
of radical inverse function
\[
\omega_b(n) = \frac{(S(b)d_0) \bmod b}{b} + \frac{(S(b)d_1) \bmod b}{b^2} + \cdots + \frac{(S(b)d_j) \bmod b}{b^{j+1}},
\]
where $S(b)$ is defined to be a number such that $S(b)/b$ is close to the
fractional part of $\sqrt{b}$.  The details are given in algorithm
\ref{alg:warnock_s_b_search}.  In \qmcpack{} this type of radical inverse function can be obtained with
\begin{lstlisting}
call phi = radical_inverse_phicf(n, b)
\end{lstlisting}

\begin{algorithm}[h!]
\begin{algorithmic}
\STATE $X_U = \lceil b\{\sqrt{b}\} \rceil$ (where $\{x\}$ denotes the fractional part of $x$)
\STATE $X_L = \lfloor b\{\sqrt{b}\} \rfloor$
\STATE $\frac{X_U}{b} = /d_1, d_2, d_3, \ldots, d_k/$ (continued fraction expansion of $\frac{X_U}{b}$)
\STATE $\frac{X_L}{b} = /e_1, e_2, e_3, \ldots, e_m/$ (continued fraction expansion of $\frac{X_L}{b}$)
\STATE Choose $S(b)$ as either $X_U$ or $X_L$ according to:
\begin{enumerate}
\item The smaller sum of partial quotients
\item The smallest largest partial quotient
\item The nearest to the fractional part of $\sqrt{b}$
\end{enumerate}
\end{algorithmic}
\caption{Warnock's algorithm to determine $S(b)$}
\label{alg:warnock_s_b_search}
\end{algorithm}

\section{Faure sequences}

If $n-1 = \sum_{j=0}^{m}d_jr^j$ is a strict positive integer, then then
the first coordinate of the $n$'th point of the Faure sequence is given by
\cite{faure82}
\[
x_n^1 = \sum_{j=0}^{m}d_jr^{-j-1}
\]
If $\vec{x}_n^1$ represents the vector $[d_0, d_1, \dots, d_m]^T$ and
$\phi_p(\vec{x})$ means taking the simple radical inverse function in base
$p$ with the
digits contained in $\vec{x}$, then the $n$'th point of the Faure sequence is
given by:
\begin{eqnarray*}
\vec{x}_n & = & (\phi_p(\vec{x}_n^1),\ \phi_p(\vec{x}_n^2),\ \phi_p(\vec{x}_n^3), \dots,\ \phi_p(\vec{x}_n^s)) \\
          & = & (\phi_p(\vec{x}_n^1),\ \phi_p(C\vec{x}_n^1),\ \phi_p(C^2\vec{x}_n^1), \dots,\ \phi_p(C^{s-1}\vec{x}_n^1)) \\
          & = & (\phi_p(\vec{x}_n^1),\ \phi_p(C\vec{x}_n^1),\ \phi_p(C\vec{x}_n^2), \dots,\ \phi_p(C\vec{x}_n^{s-1}))
\end{eqnarray*}
where $p$ is a prime greater than or equal to $s$ and the $C^j=P^j$ are
powers of the pascal matrix $P \pmod{p}$:
\[
C^j = 
P^j \bmod{p} =
\begin{bmatrix}
1 & 1 & 1 & 1 & 1 \\
0 & 1 & 2 & 3 & 4 \\
0 & 0 & 1 & 3 & 6 \\
0 & 0 & 0 & 1 & 4 \\
0 & 0 & 0 & 0 & 1
\end{bmatrix}^j \bmod{p}.
\]
For every $j \geq 1$, the $j$-th power of $C$ is the matrix
\[
C^j = (c_{uv}^{(j)})
\qquad \text{where $c_{uv}^{(j)}=j^{v-u}c_{uv}$ for all integers $u, v \geq 0$.}
\]
%
Tezuka's \verb!GFaure! \cite{tezuka95} has the $j$th dimension generator matrix as
\[
C^{(j)} = A^{(j)}P^j \bmod{p},
\]
where $A^{(j)}$ is a random nonsingular lower triangular matrix and can be
expressed as
\[
A^{(j)} =
\begin{bmatrix}
h_{11} & 0      & 0      & 0      & \ldots & 0 \\
g_{21} & h_{22} & 0      & 0      & \ldots & 0 \\
g_{31} & g_{32} & h_{33} & 0      & \ldots & 0 \\
\vdots & \vdots & \vdots & \ddots & \ddots & \vdots \\
\end{bmatrix}_{m \times m}.
\]
%
\textcolor{red}{TODO: investigate Tezuka's different scrambling types:
\begin{itemize}
\item Random linear scrambling
\item Random linear-digit scrambling
\item I-binomial scrambling
\item Striped matrix scrambling
\end{itemize}}

\section{Halton sequences}

The \emph{van der Corput} sequence in base $b$ is defined as the
one-dimensional point set $\{\phi_b(n)\}_{n=0}^{\infty}$.  Halton
\cite{halton60} extends this definition to the $s$-dimensional sequence
$\{\vec{x}_i\}$, defining
\[
\vec{x}_n = (\phi_{b_1}(n), \dots, \phi_{b_s}(n)), \qquad n = 0, 1, \dots
\]
The integers $b_1, \dots, b_n$ are greater than one and pairwise prime.  Most of the time, they are chosen as the first $s$ primes.

To be able to generate Halton points with \qmcpack{}, it must first be
initialized using
\begin{lstlisting}
call init_halton(s, startindex, myprimes, perm_type)
\end{lstlisting}
where \verb!s! is an integer indicating the dimension of the points to be
generated. \verb!startindex! and \verb!myprimes! are optional one-dimensional
arrays of length \verb!s!.  The former contains possible starting values $n$
for each dimension, while the latter contains the primes $b$ to be used
as bases for the different dimensions.  If no startindex is specified, the
default array \verb!(/ 1, ..., 1 /)! is used.  The first \verb!s! primes are
used as default values for the bases.  Also, by default, no scrambling is
applied.

After the Halton sequence has been initialized, subsequent points of it can be
generated and assigned to the one-dimensional array \verb!x! of length
\verb!s! using
\begin{lstlisting}
call next_halton(x)
\end{lstlisting}

When all the computations are done and no more Halton points need to be
generated,
\begin{lstlisting}
call free_halton()
\end{lstlisting}
is necessary in order to free up the memory allocated during the initialization
phase.

In analogy with the different types of radical inverse functionality as
described in section \ref{sec:radical_inverse_functionality}, \qmcpack{}
supports different types of scrambled Halton sequences.  Their use is
straightforward and the arguments are similar as in section
\ref{sec:radical_inverse_functionality}.  The available routines are
\begin{lstlisting}
call next_halton_scrambled(x)
call next_halton_folded(x)
call next_halton_phicf(x)
call next_halton_modified(x [, increased] [, scrambled])
\end{lstlisting}
A last type of Halton scrambling is by Mascagni and Chi \cite{mascagni04chi}.
They considered the linear scrambling $\pi_{b_i}(d_j)=w_id_j \pmod{b_i}$ and
searched for optimal $w_i$.  The values they have found are tabulated in
\cite{mascagni04chi} for the first 40 dimensions.  In \qmcpack{}, this type of
scrambling is available via
\begin{lstlisting}
call next_halton_chi_permuted(x)
\end{lstlisting}


\section{Hammersley point set}

Hammersley point sets are constructed by taking a special kind of structure for
the first dimension, and then using a Halton sequence to fill up the other
dimensions.  It appeared to us that in the literature, several methods are used
to fill up the first dimension.  For example, Niederreiter
\cite{niederreiter92}, defines the $N$-element Hammersley point set in the
bases $b_1,\dots,b_{s-1}$ as
\[
\vec{x}_n = \left(\frac{n}{N}, \phi_{b_1}(n), \dots, \phi_{b_s}(n)\right), \qquad n = 0, 1, \dots, N-1
\]
while Halton \cite{halton60} mentions that Hammersley's suggestion is to use
the $s$-dimensional sequence
\[
\vec{x}_n = \left(\frac{n}{N}, \phi_{b_1}(n), \dots, \phi_{b_s}(n)\right), \qquad n = 1, \dots, N
\]
and last but not least Fang and Wang define the Hammersley set as
\[
\vec{x}_n = \left(\frac{2n-1}{N}, \phi_{b_1}(n), \dots, \phi_{b_s}(n)\right), \qquad n = 1, \dots, N
\]
In \qmcpack{}, the Hammersley set can be generated with
\begin{lstlisting}
call hammersley(n, s [, myprimes] [, startindex] [, leap] [, def], x)
\end{lstlisting}
Here, all arguments are as usual, and \verb!def! is an extra optional argument
that specifies which definition of the Hammersley set should be used.  It can be
one of the values \verb!Halton!, \verb!Niederreiter!, \verb!Fang_Wang!

\section{Randomization}

\subsection{Cranley-Patterson rotations}

One simple method to randomize a pointset is to make use of the so called
`Cranley-Patterson' rotations.  The method consists in generating a random
vector $\vec{\Delta}$ uniformly in $[0,1)^s$ and add it to each point of the
pointset modulo 1.  So if $P_N = \{\vec{x}_i,\ i=1,\dots,N\}$ is the original
pointset, then the randomized point-set is given by
\[
\tilde{P}_N = \{ \vec{x}_i + \vec{\Delta} \bmod 1,\ i=1,\ldots,N\}.
\]
Before randomizing points or pointsets in \qmcpack{}, the random-shift vector
has to be initialized.  This can be done with
\begin{lstlisting}
call init_random_shift(s)
\end{lstlisting}
After that, there are two options.  After generating the next point of a
certain pointset, you can apply the random shiftvector to it by writing
\begin{lstlisting}
call random_shift_point(x)
\end{lstlisting}
or if you have the complete pointset available as a two-dimensional array, you
can simply randomize it via
\begin{lstlisting}
call random_shift_pointset(x)
\end{lstlisting}
Once all computations are done, the memory occupied by the random shiftvector
can be released with
\begin{lstlisting}
call free_cranley_patterson()
\end{lstlisting}
