\chapter{Measuring the quality of point sets}

\begin{itemize}
\item Explain discrepancy, and explain what kind of routines \qmcpack{} provides
      to calculate it.
\end{itemize}

For a set $J \subseteq [0,1)^s$ and a sequence $P$ of $N$ points $\vec{x}_i,\
i=1,\ldots,N$, in $[0,1)^s$, define
\begin{equation} \label{eq:local_discrepancy}
R_N(J) := \frac{1}{N}\sum{\chi_J(\vec{x}_n)}-\Vol(J),
\end{equation}
where $\chi_J$ is the characteristic function of $J$ and $\Vol(J)$ is the volume
of $J$.
If $E^*$ is the set of sub-rectangles from $[0,1)^s$ with one corner at $\vec{0}$, then the
$L_{\infty}$ star discrepancy is defined as
\[
D_N^*(P) := \sup_{J \in E^*}|R_N(J)|.
\]
If $E$ is the set of all sub-rectangles from $[0,1)^s$, then the $L_{\infty}$
extreme discrepancy is defined as
\[
D_N(P) := \sup_{J \in E}|R_N(J)|.
\]
For a one-dimensional pointset $\{x_1,\ldots,x_N\}$ with $0 \leq x_1 \leq x_2 \leq \cdots \leq x_N \leq 1$ we have that
\[
D_N^*(x_1,\ldots,x_n) = \frac{1}{2N} + \max_{1 \leq n \leq N} \left| x_n -
\frac{2n-1}{2N}\right|,
\]
and
\[
D_N(x_1,\ldots,x_n) = \frac{1}{N}
                        + \max_{1\leq n \leq N}\left(\frac{n}{N}-x_n\right)
                        - \min_{1\leq n \leq N}\left(\frac{n}{N}-x_n\right),
\]
which can be calculated in \qmcpack{} as
\begin{lstlisting}
call d_n_star1d(x, d_n_star)
\end{lstlisting}
respectively
\begin{lstlisting}
call d_n_extreme1d(x, d_n_extreme)
\end{lstlisting}
with \verb!x! the rank 1 array containing the pointset and \verb!d_n_star! and \verb!d_n_extreme! the results.

The $L_2$ star discrepancy of an $s$-dimensional point set $P$ is defined as
\[
T_N^*(P) := \left[ \int_{I^s} \left( R_N(J(\vec{x}))\right)^2 d\vec{x}\right]^\frac{1}{2},
\]
where $J(\vec{x})$ is the rectangle with opposite corners at $\vec{0}$ and
$\vec{x}$.

If $x_k^{(i)}$ denotes the $i$-th component of the
$s$-dimensional point $\vec{x}_k$ in a point set $P$, then it can be shown
\cite{warnock72} that the square of the $L_2$ star discrepancy of $P$ is given
by
\begin{equation}
\label{eq:def T_N}
[T_N^*(P)]^2 =
  \frac{1}{N^2}\sum_{k=1}^N\sum_{m=1}^N\prod_{i=1}^s\left(1-\max(x_k^{(i)},x_m^{(i)})\right)
  - \frac{2^{1-s}}{N} \sum_{k=1}^{N} \prod_{i=1}^s\left(1-{x_k^{(i)}}^2\right)
  + 3^{-s}.
\end{equation}

In QMCPACK, the \emph{square} of the $L_2$ star discrepancy can be calculated
with
\begin{lstlisting}
call T_N_star_squared(x, tn2)
\end{lstlisting}
where \verb!x! is a rank 2 array representing the pointset.  Important to notice is that $x_k^{(i)}$ must be represented as \verb|x(i,k)|, so the first
dimension of the array
\verb|x| represents the different dimensions of the pointset, while the second
dimension of \verb!x! is used for indexing the different points.
The cumulative results are stored in the rank 1 array \verb!tn2!, so
\verb!tn2(i)! contains $(T_i^*)^2$, the square of the $L_2$ star discrepancy
of the first \verb!i! points.

Similarly, if $J(\vec{x})$ can be any cube, then this is called the $L_2$
extreme discrepancy and it can be calculated via the formula (See
\cite{morokoff94caflish})
\begin{align} \label{eq:t_n_extreme}
[T_N(P)]^2 &=
  \frac{1}{N^2}\sum_{k=1}^N\sum_{m=1}^N\prod_{i=1}^s\left[1-\max(x_k^{(i)},x_m^{(i)})\right] \cdot \min(x_k^{(i)},x_m^{(i)}) \notag \\
           & \quad
  - \frac{2^{1-s}}{N} \sum_{k=1}^{N} \prod_{i=1}^s\left(1-{x_k^{(i)}}\right)x_k^{(i)} + 12^{-s}.
\end{align}
and in \qmcpack{} this is done with
\begin{lstlisting}
call T_N_extreme_squared(x, tn2)
\end{lstlisting}
Concerning the arguments, the same remarks as for the $L_2$ star discrepancy
are valid.
